\subsection{Related Platforms}
\label{sec:related_platforms}

The concept of creating local ad-hoc networks with very little configuration has been under development for several years.

Universal Plug and Play (UPnP) is one of the more widely-deployed examples of a zero-configuration networking technology in use today. 
It consists of a set of networking protocols developed to facilitate interaction with services offered by any networked device on a local network.
Since it leverages common protocols (HTTP/XML/SOAP on UDP/IP) and is agnostic to the link medium, it is truly cross-platform, extending not only to phones and laptops but also to printers, WiFi routers, and audio-visual equipment, to name a few examples.
Many consumer devices currently come with UPnP capability built in.
However, UPnP has been criticized for being insecure (by default, it assumes that all devices on the local network can be trusted, and so does not provide any means for authentication) and unscalable (due to use of multicast for service discovery).

More recently in 2017, as part of its \textit{Nearby} project, \textit{Google} released its \textit{Connections} API, which enables Android devices in close proximity to one another to communicate in a peer-to-peer fashion.\footnote{https://developers.google.com/nearby/connections/overview}
This is done over a seamless mix of Bluetooth, BLE, and WiFi hotspots.
Unfortunately, Google has only made this API available on the Android platform; we seek a solution that is truly cross-platform.

An early example of fully in-browser web server technology was Opera Unite.
This extension to the Opera browser enabled users with Opera accounts to serve applications directly from their browsers to anyone on the Web, via a URL pointing to a subdomain on Opera's proxy servers.\footnote{https://maqentaer.com/devopera-static-backup/http/dev.opera.com/articles/view/opera-unite-developer-primer-revisited/index.html}
Communication between clients and servers in this way could either be direct peer-to-peer, or via Opera Unite's proxy servers.
Though the service was popular with a sizeable subset of Opera users, especially for purposes such as file sharing and media streaming, Opera eventually retired it to consolidate the multiple extension frameworks offered in its browser.\footnote{http://www.instantfundas.com/2012/04/opera-to-discontinue-unite-widgets-and.html}

\textit{FlyWeb} is a Web API developed by the Mozilla Firefox community which enables clients of Web applications to publish a local server from within the browser.
Building on the concept of zero-configuration networks and its mDNS/DNS-SD protocols~\cite{rfc6762, rfc6763}, the server advertises itself in the local network and can be discovered by other devices which become clients to the server by connecting via a HTTP or WebSocket connection.
This essentially enables cross-device communication within a local-area network.

For our approach, we decided for FlyWeb since it is open source and entirely based on Web technology, making it widely available.
We think that the idea of hosting an ad-hoc network from within a Web application in the browser has great potential, and we think that applications for this network can benefit from graceful recovery upon disconnection of local servers.
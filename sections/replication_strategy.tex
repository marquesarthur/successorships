\subsection{Replication Strategy}

We aim to leverage the concepts of \textbf{eventual consistency} and \textbf{lazy replication} for our fault tolerance approach. 
We argue that our domain of local ad-hoc offline Web applications will mostly tolerate minor time frames of inconsistent states and we rather aim to increase availability and performance.

When a server is initialized and starts running, we envision that our API will create a state for that server and that this state is updated after the execution of any received message callback. 
As clients connect to a server, the clients themselves assign their own host names and the server keeps a list of connected clients.
Changes to the server state are broadcasted to all clients. 
A central concern, therefore, is to handle the possibility of inconsistencies in the versions of the server state maintained by each client: what if the server dies before changes are broadcasted? 
What if the broadcast to our next server fails and the next server takes over with a stale state? 

Our current idea is to converge to a consistent state in accordance with the eventual consistency principle when clients (re-)connect to the new server: upon connecting, they can send their current state associated with a timestamp of the last modification.
From this information, the most current version of the previous server's state can be determined from amongst all clients' copies, and then propagated to the new server.
This system opens up the risk of being spoofed by malicious clients imposing a manipulated state onto our network, but as stated previously, we allow ourselves to assume full trust between all participants in the network for simplification.
 
We have also considered using a CRDT for our implementation, but we believe that the constraint of commutativity of operations, or associativity of merging conflicting states, is potentially too restricting for the range of applications we would want to allow to run on our framework.

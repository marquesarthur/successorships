\documentclass[sigconf]{acmart}
\usepackage[utf8]{inputenc}

\settopmatter{printacmref=false}
\renewcommand\footnotetextcopyrightpermission[1]{} % removes footnote with conference information in first column
\pagestyle{plain}

\usepackage{url}
\usepackage{hyperref}
\hypersetup{
    colorlinks=true,
    linkcolor=blue,
    filecolor=magenta,      
    urlcolor=cyan,
}
\urlstyle{same}


\usepackage{listings}

% Enabling Javascript syntax highlight in code snippet - BEGIN 
% https://tex.stackexchange.com/questions/89574/language-option-supported-in-listings
\usepackage{color}
\definecolor{lightgray}{rgb}{.9,.9,.9}
\definecolor{darkgray}{rgb}{.4,.4,.4}
\definecolor{purple}{rgb}{0.65, 0.12, 0.82}
\definecolor{darkgreen}{rgb}{0, .64, 0}

\lstdefinelanguage{JavaScript}{
  keywords={typeof, new, true, false, catch, function, return, null, catch, switch, var, if, in, while, do, else, case, break},
  keywordstyle=\color{blue}\bfseries,
  ndkeywords={class, export, boolean, throw, implements, import, this},
  ndkeywordstyle=\color{darkgray}\bfseries,
  identifierstyle=\color{black},
  sensitive=false,
  comment=[l]{//},
  morecomment=[s]{/*}{*/},
  commentstyle=\color{purple}\ttfamily,
  stringstyle=\color{darkgreen}\ttfamily,
  morestring=[b]',
  morestring=[b]"
}

\lstset{
   language=JavaScript,
   extendedchars=true,
   basicstyle=\footnotesize\ttfamily,
   showstringspaces=false,
   showspaces=false,
   numbers=left,
   numberstyle=\footnotesize,
   numbersep=9pt,
   tabsize=2,
   breaklines=true,
   showtabs=false,
   captionpos=b
}
% Enabling Javascript syntax highlight in code snippet - END

\newcommand{\APIName}{Successorships}

\newcommand{\APIshort}{ships}

%opening
\title{Who is the next server? Enabling fault-tolerant local Webapps}

\author{Arthur Marques \qquad Felix Grund \qquad Paul Cernek}
\affiliation{
    \institution{University of British Columbia}
    \city{Vancouver} 
    \state{BC} 
  }

\begin{document}

\maketitle

\begin{abstract}
In the face of the growing domain of the Internet of Things, we are recently observing increased usage of applications that communicate with each other within the local network. This trend has been driven by the concept of zero-configuration networking that eliminates the burden of setup procedures. While applications native to specific operating systems and devices have traditionally been the common case, browser vendors are now trying to bring this idea to the Web: any Web application can start its own server and advertise a service in the local network. The service can then be discovered and used by any connected browser-capable device in the same network which provides offline availablility and privacy and saves traffic. Since in such scenarios, any device can become a Web server, we argue that such applications can provide good fault-tolerance capabilities because the server role can simply be transferred to a client if the application's state is replicated. In this paper, we introduce a JavaScript library for such zero-configuration Web applications that enables graceful recovery after server failure by handing over the server role to the "next" client. We evaluated our approach with a sample application and usage by 5 participants. The results show that our library recovers the system gracefully after an average of 3.5 seconds and that application state is well-preserved.
\end{abstract}

\section{Introduction}
\label{sec:introduction}


% Introduce client and server model over the Internet
The client-server model describes a scenario in which one or many clients require a service or resources from a centralized server. Tacit in this model is the assumption that the server has access to resources that are unavailable to the client, whether this be compute power, storage, sensitive data, etc. 


% Introduce devices with lower computational power that can work as servers for some applications
Regarding the growing domain of the Internet of things (IoT), the proliferation of ``smart'' devices with lower constraints on computational power, storage, or sensitive data, opened new doors to a set of applications that relax some of the assumptions in client-server model. For instance, a queue application used by a TA during office hours: the TA spins up the application on her phone, and as students enter the room, they connect to the server on the TA's phone and request to be enqueued. In this example, however, we need to be explicit about the gains obtained from adhering to this client-server model, rather than implementing this application as a peer-to-peer application.


% Introduce zeroconf networks and device/service discovery
In the aforementioned example, we assume that {\it (1)} devices are in a local-area network and {\it (2)} client-devices are able to discover the server-device and its offered service in that network. Such assumptions may be satisfied in a local-area network using existing IP-based routing. Nonetheless, adding and configuring devices in a local-area network may be error prone, time consuming, or ill suited for the users of these devices. Therefore automatic approaches to identify devices [and their services] in a local-area network, such as Zero-configuration networking, are desired. 


% Introduce fly web
The suite of protocols in Zero-configuration networking (mDNS/ DNS-SD)~\cite{rfc6762,rfc6763} provides {\it (i)} the automatic assignment of IP addresses and host naming (mDNS), and {\it (ii)} service discovery (DNS-SD). Once a device has an assigned name/IP and other devices can discover this device provided services, they can use application layer protocols and communicate with each other. As an example, the Mozilla Firefox\footnote{\url{https://www.mozilla.org/en-US/foundation/}} FlyWeb\footnote{\url{https://wiki.mozilla.org/FlyWeb}} extension leverages this suite of protocols to allow clients of Web applications to start their own local Web server from within the browser. This server is then advertised in the local network such that other devices in the network can detect the new server and can connect to it via the browser. 


% Discuss fault tolerance
The FlyWeb extension caught our attention as it offers a range of new possibilities for Web applications and local vs. global network behavior. However, its current implementation has a severe limitation, the complete lack of fault tolerance. When the local server dies, the client-server network dies with it. Since this technology is inherently driven by the idea of any device being able to become a server, we assume that it is more likely that servers misbehave in comparison to the  ``traditional'' server model. 



We therefore regard fault tolerance as very important in such networks and we think that the lack of mechanism for graceful recovery from server abruptly disconnections is problematic.
For instance, in our queue example, maybe the TA needs to leave the room momentarily, and therefore has to leave the local network. In this event, we wish for the entire application state, and even the ability of new students to enqueue as they arrive, to persist even as the initial server host leaves the network (``distributed failover''). When the TA returns, the application seamlessly returns to being hosted on her device (``failback''). 


% State our project proposal
Considering the aforementioned discussion, {\bf we propose an approach to add fault tolerance to FlyWeb}. We hypothesize that a technology like FlyWeb is a good basis for fault tolerance through replication, since any participant can become the server. In the situation of a failing server, we think it is intuitive that a client can become the ``next" server and all other clients establish a connection to the new server. Obviously, replication comes at the cost of complexity. We intend to analyze the different replication strategies and choose one that adheres best to our scenario. We aim to deliver our approach in a JavaScript library -- named {\texttt{\APIName{}}} -- that provides fault tolerance to FlyWeb application developers without exposing the underlying technical details.



% Final considerations:: I'm not sure if we need this paragraph
As a final remark, it is important to emphasize that it is neither our intent to shoe-horning the client-server application model into working as a peer-to-peer service nor assume that all web applications can use our proposed approach. But we imagine some concrete use cases, such as the TA queue system, that would benefit from this fault tolerant behavior.

\section{Background}
\label{sec:background}

\subsection{Flyweb}
\label{sec:flyweb}

\subsection{Zero-configuration Networks}
\label{sec:zeroconf}

Zero-configuration networking is a combination of protocols that aim to automatically discover computers or peripherals in a network without any central servers or human administration. Zero-configuration networks have two major components that provide {\it (i)} automatic assignment of IP addresses and host naming (mDNS), and {\it (ii)} service discovery (DNS-SD).

When a device enters the local network, it assigns an IP/name pair to itself and  multicasts this pair to the local network, resolving any name conflicts that may occur in the process. IP assignment considers the link-local domain address which draws addresses from the IPv4 169.254/16 prefix and, once an IP address is selected, a host name with the suffix ``.local'' is mapped to that IP~\cite{rfc6762}. As devices are mapped to IPs/host names, their available services are discovered using a combination of DNS PTR, SRV, and TXT records~\cite{rfc6763}; their services can then be requested by other devices.

The design focus of zero-configuration networking protocols is smooth assignment of names and discovery services without the configuration tasks normally present in network infrastructure. 
Our aim is to build upon the appealing features of zero-configuration networks by making them fault tolerant.
This property is desirable in any scenario where network services need to remain stable even when the server node disconnects.
In the zero-configuration setting, we anticipate that the hassle-free set-up will be complemented naturally by a system that enables a service to be provided in uninterrupted fashion, even when the server node disconnects.


\subsection{Replication}
\label{sec:replication}

Fault tolerance and reliability in distributed systems with client-server architecture are generally achieved by data replication: information is shared on redundant server replicas such that any replica can become the new master if the current master fails. While improving system artifacts like fault-tolerance, reliability and availability, replication can come at the cost of performance: depending on the required operations in the system for replication, system performance can suffer significant bottlenecks. Different models of replication have been proposed to trade consistency for performance, resulting in different levels of consistency as a design choice for the target system.

\textbf{Active vs. passive replication.} Traditionally, two strategies of replication are distinguished: \textit{active replication} and \textit{passive replication}. In \textbf{active replication} (also called \textit{primary-backup} or \textit{master-slave}), requests to the master replica are processed to all other replicas. Given the same initial state and request sequence, all replicas will produce the same response sequence and reach the same final state. Active replication has been become most prominent with the introduction of the State Machine Replication model which was introduced in the 1980s \cite{Lamport:1984} and later refined in \cite{Schneider:1990}. It is based on the concept of distributed consensus with the goal of reliably reaching a stable state of the system in the presence of failures. While providing small recovery delay after failures due to an imposed total order of state updates, computation performance can suffer tremendous bottlenecks since updates must be sequentially propagated through all replicas. The second strategy of \textbf{passive replication} (also called multi-primary or multi-master scheme) potentially improves computation performance by relaxing sequential ordering: clients communicate with a master replica and updates are forwarded to backup replicas. Computation performance is improved with this pattern since all computation takes place on the master replica and only the results are propagated. The downside of the approach is that more network bandwidth is required if updates are large. Since the primary replica represents a single point of entry to clients with this approach, there must be some kind of distributed concurrency control in order to reliably restore state when the primary fails. This makes the implementation of this approach more complex and recovery potentially slower.

\textbf{Lazy replication.} A third strategy of replication was proposed in 1990: \textit{lazy replication} \cite{Ladin:1990,Ladin:1992} (also called \textit{optimistic replication}) aims at providing highest possible performance by sacrificing consistency significantly. With this approach, replicas periodically exchange information, tolerating out-of-sync periods but guarantee to catch up eventually. While the traditional approaches guarantee from the beginning that all replicas have the exact same state at any point in time, lazy replication allows states to diverge on replicas, but guarantees that the states converge when the system quiesces for some time period. In contrast to the strong consistency models used in the traditional approaches, lazy replication is based on eventual consistency which has gained more attention recently, in particular with the introduction of \textit{conflict-free replicated data types} \cite{Shapiro:2011}, online editing platforms and NoSQL cloud databases \footnote{https://cloud.google.com/datastore/docs/articles/balancing-strong-and-eventual-consistency-with-google-cloud-datastore/} which rely on immediate response for general usability.

\textbf{Eventual consistency and conflict-free replicated data types.}

 




\section{Proposed Approach}
\label{sec:approach}

\section{Evaluation}
\label{sec:evaluation}

Due to the nature of our project, i.e. an offline fault-tolerant client-server Web browser API, our evaluation will be twofold: \textit{(1)} comparing the network traffic induced by a Web app built on our framework, to a traditional client-server version; and \textit{(2)} evaluating the robustness to failure of our framework.

\subsection{Traffic comparison}

First, we want to measure network traffic in this offline network and compare it against a traditional client-server Web application. 
Despite having different network characteristics (local-area vs Internet), this comparison will help us to identify and discuss possible benefits and drawbacks of our approach.

We will monitor traffic at the packet level using an implementation of the \texttt{pcap} API to measure network traffic.
Traffic will be measured for an application running in {\it (i)} a traditional client-server architecture, {\it (ii)} a default FlyWeb implementation, and also {\it (iii)} in our fault-tolerant API. 
We aim to compare traffic in each one of these scenarios and discuss their differences. 
For the traditional client-server vs. FlyWeb comparison, our purpose is to evaluate differences in delay, while for the within-FlyWeb comparison, we want to compare network overhead generated by our fault-tolerance strategy. 

\subsection{Robustness to failures}

Second, we want to measure network traffic in the face of failures.
How much network traffic is required to achieve stability once a server device fails? 
How long does it take for this stability to be achieved?
Is our approach to fault-tolerance scalable? 
These are some of the questions that we want to answer with the second evaluation.

As for our second evaluation, we will write scripts that simulate client connecting to a server device through our \texttt{\APIName} API. 
Once a set of clients establish communication, our simulation will then remove the server device from the network such that we can evaluate how \texttt{\APIName} handles failures. 

Regarding our simulations, we will consider a queue system as a baseline application.
The queue will have $i$ configurable consumers, and $j$ configurable producers which will produce-consume queue entries at random given times. 
Such entries will also have a configurable payload size. 
Such system will give us a local-area network with $n = i + j$ devices, and it will also allow us to experiment with different size configurations. 
For instance, we can compare scenarios with a small, medium, or large number of devices connected.


\section{Timeline}
\label{sec:timeline}

We consider late November as the project final deadline. With that in mind, there are a few key activities that we define as milestones, such that we keep up with the project schedule.

\begin{itemize}
    \item {\bf Oct 25th: } Study and evaluate replication patterns. Build a sample FlyWeb queue application;
    \item {\bf Nov 1st: } Implement the core functionalities of the \APIName{} API.
    \item {\bf Nov 8th: } Continue API implementation. Start writing scripts for evaluation;
    \item {\bf Nov 22nd: } Wrap-up API. Write scripts to analyze and plot data; Start drafting project report.
    \item {\bf Nov 30th: } Supposed project deadline?
\end{itemize}

\bibliographystyle{abbrv}
\bibliography{flyweb_paper}

\end{document}


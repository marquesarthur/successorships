\documentclass[sigconf]{acmart}
\usepackage[utf8]{inputenc}

\settopmatter{printacmref=false}
\renewcommand\footnotetextcopyrightpermission[1]{} % removes footnote with conference information in first column
\pagestyle{plain}

\usepackage{url}
\usepackage{hyperref}
\hypersetup{
    colorlinks=true,
    linkcolor=blue,
    filecolor=magenta,      
    urlcolor=cyan,
}
\urlstyle{same}


\usepackage{listings}
\usepackage{lipsum}

% Enabling Javascript syntax highlight in code snippet - BEGIN 
% https://tex.stackexchange.com/questions/89574/language-option-supported-in-listings
\usepackage{color}
\definecolor{lightgray}{rgb}{.9,.9,.9}
\definecolor{darkgray}{rgb}{.4,.4,.4}
\definecolor{purple}{rgb}{0.65, 0.12, 0.82}
\definecolor{darkgreen}{rgb}{0, .64, 0}

\lstdefinelanguage{JavaScript}{
  keywords={typeof, new, true, false, catch, function, return, null, catch, switch, var, if, in, while, do, else, case, break},
  keywordstyle=\color{blue}\bfseries,
  ndkeywords={class, export, boolean, throw, implements, import, this},
  ndkeywordstyle=\color{darkgray}\bfseries,
  identifierstyle=\color{black},
  sensitive=false,
  comment=[l]{//},
  morecomment=[s]{/*}{*/},
  commentstyle=\color{purple}\^amily,
  stringstyle=\color{darkgreen}\ttfamily,
  morestring=[b]',
  morestring=[b]"
}

\lstset{
   extendedchars=true,
   basicstyle=\footnotesize\ttfamily,
   showstringspaces=false,
   showspaces=false,
   tabsize=2,
   breaklines=true,
   showtabs=false,
   captionpos=b,
   frame=single,
   xleftmargin=0.5em,
   belowcaptionskip=0em
}

\setlength{\belowcaptionskip}{-1em}
% Enabling Javascript syntax highlight in code snippet - END

\newcommand{\APIName}{Successorships }
\newcommand{\APINameNoSpace}{Successorships}

\newcommand{\APIshort}{Shippy }
\newcommand{\rpm}{\raisebox{.2ex}{$\scriptstyle\pm$}}

\newcommand{\accessed}{accessed 2017-12-11}

\usepackage{array}
\newcolumntype{L}[1]{>{\raggedright\let\newline\\\arraybackslash\hspace{0pt}}m{#1}}
\newcolumntype{C}[1]{>{\centering\let\newline\\\arraybackslash\hspace{0pt}}m{#1}}
\newcolumntype{R}[1]{>{\raggedleft\let\newline\\\arraybackslash\hspace{0pt}}m{#1}}


\graphicspath{{figures/}{pictures/}{images/}{./}}

%opening
\title{Successorships - Fault-tolerant local WebApps}

\author{Arthur Marques \qquad Felix Grund \qquad Paul Cernek}
\affiliation{
    \institution{University of British Columbia}
    \city{Vancouver} 
    \state{BC} 
  }

\begin{document}

\begin{abstract}
As networking capabilities become more ubiquitous across different types of devices, applications that communicate over local area networks are becoming increasingly common.
This trend has been accelerated by the adoption of zero-configuration (zeroconf) networking standards that eliminate the burden of setup procedures.
Applications operating in zeroconf settings often face the challenge of maintaining reliability and consistency in the face of wireless links and mobile devices, resulting in potentially intermittent connectivity between hosts.
Fault-tolerance is thus a desirable property for such applications, but it can be difficult to achieve in practice.
To this end, we introduce Successorships, a JavaScript library that provides the appealing qualities of zeroconf enriched with fault-tolerance, to applications running in the browser.
Our library enables graceful recovery after server failure by handing over the server role to one of the clients currently in the network.
We evaluate our approach with a sample application, focusing on usage scenarios that involve interactions between users walking in and out of a single room. 
We find that our library recovers from failures gracefully after an average of 20 seconds, and that application state is maintained with eventual consistency.
\end{abstract}

\maketitle

\section{Introduction}
\label{sec:introduction}


% Introduce client and server model over the Internet
The client-server model describes a scenario in which one or many clients require a service or resources from a centralized server. Tacit in this model is the assumption that the server has access to resources that are unavailable to the client, whether this be compute power, storage, sensitive data, etc. 


% Introduce devices with lower computational power that can work as servers for some applications
Regarding the growing domain of the Internet of things (IoT), the proliferation of ``smart'' devices with lower constraints on computational power, storage, or sensitive data, opened new doors to a set of applications that relax some of the assumptions in client-server model. For instance, a queue application used by a TA during office hours: the TA spins up the application on her phone, and as students enter the room, they connect to the server on the TA's phone and request to be enqueued. In this example, however, we need to be explicit about the gains obtained from adhering to this client-server model, rather than implementing this application as a peer-to-peer application.


% Introduce zeroconf networks and device/service discovery
In the aforementioned example, we assume that {\it (1)} devices are in a local-area network and {\it (2)} client-devices are able to discover the server-device and its offered service in that network. Such assumptions may be satisfied in a local-area network using existing IP-based routing. Nonetheless, adding and configuring devices in a local-area network may be error prone, time consuming, or ill suited for the users of these devices. Therefore automatic approaches to identify devices [and their services] in a local-area network, such as Zero-configuration networking, are desired. 


% Introduce fly web
The suite of protocols in Zero-configuration networking (mDNS/ DNS-SD)~\cite{rfc6762,rfc6763} provides {\it (i)} the automatic assignment of IP addresses and host naming (mDNS), and {\it (ii)} service discovery (DNS-SD). Once a device has an assigned name/IP and other devices can discover this device provided services, they can use application layer protocols and communicate with each other. As an example, the Mozilla Firefox\footnote{\url{https://www.mozilla.org/en-US/foundation/}} FlyWeb\footnote{\url{https://wiki.mozilla.org/FlyWeb}} extension leverages this suite of protocols to allow clients of Web applications to start their own local Web server from within the browser. This server is then advertised in the local network such that other devices in the network can detect the new server and can connect to it via the browser. 


% Discuss fault tolerance
The FlyWeb extension caught our attention as it offers a range of new possibilities for Web applications and local vs. global network behavior. However, its current implementation has a severe limitation, the complete lack of fault tolerance. When the local server dies, the client-server network dies with it. Since this technology is inherently driven by the idea of any device being able to become a server, we assume that it is more likely that servers misbehave in comparison to the  ``traditional'' server model. 



We therefore regard fault tolerance as very important in such networks and we think that the lack of mechanism for graceful recovery from server abruptly disconnections is problematic.
For instance, in our queue example, maybe the TA needs to leave the room momentarily, and therefore has to leave the local network. In this event, we wish for the entire application state, and even the ability of new students to enqueue as they arrive, to persist even as the initial server host leaves the network (``distributed failover''). When the TA returns, the application seamlessly returns to being hosted on her device (``failback''). 


% State our project proposal
Considering the aforementioned discussion, {\bf we propose an approach to add fault tolerance to FlyWeb}. We hypothesize that a technology like FlyWeb is a good basis for fault tolerance through replication, since any participant can become the server. In the situation of a failing server, we think it is intuitive that a client can become the ``next" server and all other clients establish a connection to the new server. Obviously, replication comes at the cost of complexity. We intend to analyze the different replication strategies and choose one that adheres best to our scenario. We aim to deliver our approach in a JavaScript library -- named {\texttt{\APIName{}}} -- that provides fault tolerance to FlyWeb application developers without exposing the underlying technical details.



% Final considerations:: I'm not sure if we need this paragraph
As a final remark, it is important to emphasize that it is neither our intent to shoe-horning the client-server application model into working as a peer-to-peer service nor assume that all web applications can use our proposed approach. But we imagine some concrete use cases, such as the TA queue system, that would benefit from this fault tolerant behavior.

\section{Background}
\label{sec:background}

\subsection{Flyweb}
\label{sec:flyweb}

\subsection{Zero-configuration Networks}
\label{sec:zeroconf}

Zero-configuration networking is a combination of protocols that aim to automatically discover computers or peripherals in a network without any central servers or human administration. Zero-configuration networks have two major components that provide {\it (i)} automatic assignment of IP addresses and host naming (mDNS), and {\it (ii)} service discovery (DNS-SD).

When a device enters the local network, it assigns an IP/name pair to itself and  multicasts this pair to the local network, resolving any name conflicts that may occur in the process. IP assignment considers the link-local domain address which draws addresses from the IPv4 169.254/16 prefix and, once an IP address is selected, a host name with the suffix ``.local'' is mapped to that IP~\cite{rfc6762}. As devices are mapped to IPs/host names, their available services are discovered using a combination of DNS PTR, SRV, and TXT records~\cite{rfc6763}; their services can then be requested by other devices.

The design focus of zero-configuration networking protocols is smooth assignment of names and discovery services without the configuration tasks normally present in network infrastructure. 
Our aim is to build upon the appealing features of zero-configuration networks by making them fault tolerant.
This property is desirable in any scenario where network services need to remain stable even when the server node disconnects.
In the zero-configuration setting, we anticipate that the hassle-free set-up will be complemented naturally by a system that enables a service to be provided in uninterrupted fashion, even when the server node disconnects.


\subsection{Replication}
\label{sec:replication}

Fault tolerance and reliability in distributed systems with client-server architecture are generally achieved by data replication: information is shared on redundant server replicas such that any replica can become the new master if the current master fails. While improving system artifacts like fault-tolerance, reliability and availability, replication can come at the cost of performance: depending on the required operations in the system for replication, system performance can suffer significant bottlenecks. Different models of replication have been proposed to trade consistency for performance, resulting in different levels of consistency as a design choice for the target system.

\textbf{Active vs. passive replication.} Traditionally, two strategies of replication are distinguished: \textit{active replication} and \textit{passive replication}. In \textbf{active replication} (also called \textit{primary-backup} or \textit{master-slave}), requests to the master replica are processed to all other replicas. Given the same initial state and request sequence, all replicas will produce the same response sequence and reach the same final state. Active replication has been become most prominent with the introduction of the State Machine Replication model which was introduced in the 1980s \cite{Lamport:1984} and later refined in \cite{Schneider:1990}. It is based on the concept of distributed consensus with the goal of reliably reaching a stable state of the system in the presence of failures. While providing small recovery delay after failures due to an imposed total order of state updates, computation performance can suffer tremendous bottlenecks since updates must be sequentially propagated through all replicas. The second strategy of \textbf{passive replication} (also called multi-primary or multi-master scheme) potentially improves computation performance by relaxing sequential ordering: clients communicate with a master replica and updates are forwarded to backup replicas. Computation performance is improved with this pattern since all computation takes place on the master replica and only the results are propagated. The downside of the approach is that more network bandwidth is required if updates are large. Since the primary replica represents a single point of entry to clients with this approach, there must be some kind of distributed concurrency control in order to reliably restore state when the primary fails. This makes the implementation of this approach more complex and recovery potentially slower.

\textbf{Lazy replication.} A third strategy of replication was proposed in 1990: \textit{lazy replication} \cite{Ladin:1990,Ladin:1992} (also called \textit{optimistic replication}) aims at providing highest possible performance by sacrificing consistency significantly. With this approach, replicas periodically exchange information, tolerating out-of-sync periods but guarantee to catch up eventually. While the traditional approaches guarantee from the beginning that all replicas have the exact same state at any point in time, lazy replication allows states to diverge on replicas, but guarantees that the states converge when the system quiesces for some time period. In contrast to the strong consistency models used in the traditional approaches, lazy replication is based on eventual consistency which has gained more attention recently, in particular with the introduction of \textit{conflict-free replicated data types} \cite{Shapiro:2011}, online editing platforms and NoSQL cloud databases \footnote{https://cloud.google.com/datastore/docs/articles/balancing-strong-and-eventual-consistency-with-google-cloud-datastore/} which rely on immediate response for general usability.

\textbf{Eventual consistency and conflict-free replicated data types.}

 




%\section{Motivating Example}
\label{sec:motivating_example}

\section{Successorships}
\label{sec:approach}

Our goal is to provide a framework to build Zeroconf Web applications and expose an easy-to-use API to application developers. 
We aim to provide fault-tolerance seamlessly without the developer having to deal with the underlying details of state replication and consistency challenges. 
We implemented our approach as a JavaScript library and describe it in the following sections.
We first cover a few necessary assumptions to make this implementation tractable and then explain our exposed API. 
We then continue with a more detailed conceptual description and our replication strategy and consistency guarantees.
Finally, we elaborate on what failure scenarios are handled by our approach.

\input{subsections/approach_assumptions}

\subsection{API Overview}
\label{sub:approach_api_overview}

Successorships provides a framework for fault-tolerant local area Web applications.
Its API is designed to hide underlying details of state replication and the distribution of client and server roles.
The goal of this design is to spare the application developer the complexity of network behavior and let her focus on the implementation details of the application itself.
The library is shipped as a JavaScript file {\ttfamily shippy.js} to be included in HTML files of the Web application.
When loaded, all functionality is exposed on a JavaScript object {\ttfamily Shippy} that resides in the browser's global {\ttfamily window} object.
This object is the only place of interference with the browser's global namespace to avoid naming collisions with the application's environment.

\subsection{Conceptual Description}
\label{sub:approach_conceptual_description}

\textbf{Architecture.} Figure~\ref{fig:stack} shows the \APIName stack.
On the bottom layer are the \textit{Application layer protocols}.
The Zeroconf protocols \textit{mDNS} and \textit{DNS-SD} are used for the advertisement and discovery of \APIshort services in the local network\footnote{We do not use these protocols directly but rely on their implementation in \textit{FlyWeb}.}. 
In terms of application-level network communication we use the common protocols of \textit{WebSocket} and \textit{HTTP}. 
The initial request to a \APIshort app is in form of a \textit{HTTP GET} request and associated requests for static resources.  Clients then establish a stable \textit{WebSocket} channel that is used for further communication between nodes (e.g. state replication). 
Above the application layer protocols is the \textit{FlyWeb} layer. 
We use FlyWeb as a library for facilitating interaction with the Zeroconf protocols. 
\APIName was designed with as few connection points with FlyWeb as possible such that this dependency can be easily replaced with a different implementation in the future. 
Above the FlyWeb layer is the \textit{\APIName} framework that exposes the API described in section~\ref{sub:approach_api_overview} to its applications that are located in the topmost layer.

\begin{figure}[h]
    \centering
    \includegraphics[width=\linewidth]{stack}
    \caption{Successorships Stack}
    \label{fig:stack}
\end{figure}

\noindent\textbf{Roles and shared state.} When a Shippy app is loaded in the browser the new node becomes either a client or server node. If it becomes a server node, it becomes a client node to ``itself'' shortly after. The application's global state is replicated and shared among all client nodes (see later paragraphs in this section for details). The global state can contain arbitrary application data and global metadata accessible only by the \APIshort library. One such required metadata field is a {\ttfamily successors} list containing a list of current clients, except the client node that is currently also the server. This list is used to determine which node should become the next server node upon failure of the current one. Figure~\ref{fig:roles} visualizes the distribution of roles. Node \textit{A} is currently the server with nodes \textit{B...n} being clients. The list of successors is shared in the global state accessible by all nodes.

\begin{figure}[h]
    \centering
    \includegraphics[width=\linewidth]{roles}
    \caption{Successorships Roles}
    \label{fig:roles}
\end{figure}

\noindent\textbf{Service discovery.} 
The \APIshort library described in section~\ref{sec:sub:approach_api_overview} comes with a compulsory Firefox add-on responsible for notifying apps with the current set of available \APIshort services. 
In the add-on, we register an event listener at a FlyWeb component {\ttfamily FlyWebDiscoveryManager} which is only available in the permission context of add-ons, making this a required mediator between FlyWeb and \APIshort Web apps. 
The {\ttfamily FlyWebDiscoveryManager} module dispatches a list of current local services in frequent intervals using the \textit{DNS-SD} protocol described in section~\ref{sub:background_zeroconf_networking}.
We sample these events to a maximum frequency of 100ms\footnote{We observed many duplicate and overly frequent event triggers from the {\ttfamily FlyWebDiscoveryManager} making this sampling necessary.} and filter out all services that were not published in the FlyWeb context\footnote{We aim to filter this list to contain only services published with Shippy rather than FlyWeb in the future, as described in section~\ref{sec:limitations_and_future_work}.}. We then dispatch our own events containing the current list of FlyWeb services with the service name, IP address and port on the Browser's global {\ttfamily window} object. 
These events are then accessible by any \APIshort app by registering an event listener as shown in listing~\ref{lst:service_discovery}.

\begin{lstlisting}[caption={Event listener for service discovery},label={lst:service_discovery}]
window.addEventListener('flywebServicesChanged', function(event) {
    let services = event.detail.services;
    // e.g. [{ serviceName: "QueueApp",
    // serviceUrl: "http://206.12.69.249:51629" }]
});
\end{lstlisting}

\noindent\textbf{Service publication.}

\noindent\textbf{Client succession.}

\noindent\textbf{State replication.}

\noindent\textbf{Handled fault scenarios.}

%Client Succession
%State Replication

% LIMITATIONS:
% all services are published to web apps
% ip + port => no hostnames
% only mac
% time to recovery
% weak consistency

\subsection{Replication Strategy and Consistency Guarantees}
\label{sub:approach_replication_strategy}

\textbf{State replication.}

\noindent\textbf{Reaching consensus.}

\noindent\textbf{Handled fault scenarios.}

\subsection{Handled Failure Scenarios}
\label{sub:approach_handled_failure_scenarios}

With our replication and consistency strategies, we handle different scenarios of failures gracefully. There are further scenarios that represent edge cases that we do not cope with at the moment. We address these in section~\ref{sec:limitations_and_future_work}.

\textbf{Client disconnection.} While disconnecting clients are mostly a trivial circumstance in traditional Web apps, they are more complex in \APIshort apps because the system relies on clients being represented in the shared \texttt{successors} list. A disconnecting client will trigger a \texttt{close} event on the current server for the associated WebSocket channel and the disconnected client's ID is obtained. The ID is then removed from the \texttt{successors} list on the shared state and the update is broadcasted. If the disconnected client was the first element in the \texttt{successors} list, the list will be shifted and the second element will be the new first successor.

\textbf{Server disconnection.} A disconnected server will result in a WebSocket \texttt{close} event triggered at each connected client. After some time, the service discovery mechanism described in section~\ref{sub:approach_conceptual_description} will reveal that there is no more service currently registered with the app's service name. This is the time when one of the previously connected client nodes will become the next server\footnote{We considered starting the next server immediately subsequent to a WebSocket close event but restrained from this approach. See section~\ref{sec:limitations_and_future_work} for our reasoning.}: each client compares the first item in the shared \texttt{successors} list with its ID (obtained with the \texttt{welcome} message of the initial handshake). If it matches, the client will publish a server as described in section~\ref{sub:approach_conceptual_description}. Otherwise, it will continue listening for service discovery events until a service with the app's name is available and connect to it.

\textbf{Simultaneous disconnection.} In this scenario, the server and one or multiple clients fail within a short time frame. For example, both the server and the first successor could disconnect and other clients would not receive the update that the first successor disconnected. If we did not target this case specifically, no client would become the new server because they all expect a disconnected client to do so. We solve this problem with a \textit{repeated timer} algorithm: clients that are not the first item in the \texttt{successor} list set a timer that will prune the first item from the list after some time. This interval is set sufficiently to give the current first successor enough time to publish the service. The interval is varied with a random number to make it highly unlikely that two clients prune successors at about the same time which could result in race conditions publishing a service. At some point in this process, one other client will be the first in the \texttt{successors} list and will become a server. Other clients will be notified eventually by a service discovery event and connect to the new server.

\section{Evaluation}
\label{sec:evaluation}

Due to the nature of our project, i.e. an offline fault-tolerant client-server Web browser API, our evaluation will be twofold: \textit{(1)} comparing the network traffic induced by a Web app built on our framework, to a traditional client-server version; and \textit{(2)} evaluating the robustness to failure of our framework.

\subsection{Traffic comparison}

First, we want to measure network traffic in this offline network and compare it against a traditional client-server Web application. 
Despite having different network characteristics (local-area vs Internet), this comparison will help us to identify and discuss possible benefits and drawbacks of our approach.

We will monitor traffic at the packet level using an implementation of the \texttt{pcap} API to measure network traffic.
Traffic will be measured for an application running in {\it (i)} a traditional client-server architecture, {\it (ii)} a default FlyWeb implementation, and also {\it (iii)} in our fault-tolerant API. 
We aim to compare traffic in each one of these scenarios and discuss their differences. 
For the traditional client-server vs. FlyWeb comparison, our purpose is to evaluate differences in delay, while for the within-FlyWeb comparison, we want to compare network overhead generated by our fault-tolerance strategy. 

\subsection{Robustness to failures}

Second, we want to measure network traffic in the face of failures.
How much network traffic is required to achieve stability once a server device fails? 
How long does it take for this stability to be achieved?
Is our approach to fault-tolerance scalable? 
These are some of the questions that we want to answer with the second evaluation.

As for our second evaluation, we will write scripts that simulate client connecting to a server device through our \texttt{\APIName} API. 
Once a set of clients establish communication, our simulation will then remove the server device from the network such that we can evaluate how \texttt{\APIName} handles failures. 

Regarding our simulations, we will consider a queue system as a baseline application.
The queue will have $i$ configurable consumers, and $j$ configurable producers which will produce-consume queue entries at random given times. 
Such entries will also have a configurable payload size. 
Such system will give us a local-area network with $n = i + j$ devices, and it will also allow us to experiment with different size configurations. 
For instance, we can compare scenarios with a small, medium, or large number of devices connected.


\section{Limitations and Future Work}
\label{sec:limitations_and_future_work}


% ===
% Felix: below is just a random collection of notes I took during writing other parts

% LIMITATIONS:
% all services are published to web apps
% ip + port => no hostnames
% only mac
% time to recovery
% weak consistency
% server: prevent second fetch 

% LIMITATION
% sometimes broadcast whole state
% async state updates => could wait for first succ's ack

%In the current implementation, the full state is broadcasted when a new client connects. This could be improved in the future by broadcasting only a 

% LIMITATIONS
%In certain failure cases where a client state is newer than a server state, clients can essentially \textit{overwrite} the server state.
%This opens up the risk of being spoofed by malicious clients imposing a manipulated state onto our network, but as stated previously, we allow ourselves to assume full trust between all participants in the network.
%We have considered using a \textit{CRDT} for our implementation, but we believe that the constraint of commutativity of operations, or associativity of merging conflicting states, is potentially too restricting for the range of applications we would want to allow to run on our framework.
%We have also considered to notify the first client in the \texttt{successors} list and only broadcast to other clients when an acknowledgement was received.
%We faced technical difficulties with this strategy and were not convinced enough of the benefits to justify further efforts.

% LIMITATIONS
% transmission queue

% We considered starting the next server immediately subsequent to a WebSocket close event, but restrained from this approach being too error-prone. For example, a simple page reload on the current server would trigger a WebSocket \texttt{close} event on all clients and one of them immediately becoming the next server. This would introduce another race condition between the first successor and the original server that registers itself again immediately after page reload. Our existing approach deals with this problem much more gracefully: the original server will simply remain the server because the set of current services is updated only in much longer intervals. The downside of our approach is a significantly increased time of recovery.

\section{Related Work}
\label{sec:related_work}

\textbf{Zeroconf.}
Several researchers have investigated Zeroconf~\cite{Gunes2002, Bohnenkamp2003, Jara:2012:IPv6DNS-SD}.
For instance, \cite{hong2007accelerating} discusses that Zeroconf service discovery may cause overhead to the network while discovering new services.
Thus, they propose an algorithm to accelerate service discovery based on network topology changes.
Since a server failure would imply a change in the network topology, i.e. the server node being removed from the topology, 
client nodes in our implementation could use their approach to accelerate service discovery. 
Nonetheless, their implementation uses Linux Wireless extensions, which may not be accessible within a browser.
Thus, we used the FlyWeb service discovery implementation.

In~\cite{stolikj2016context}, Stolikj et al. argue that the number of published services in a network may also slow down service discovery.
As a solution, they propose a context-based approach, where queries specify which services they are interested in.
This approach is highly suitable for our work and we could use it to filter \APIshort services.
However, their approach changes service discovery queries and we question whether this could be easily integrated with existing devices.


\textbf{Local Networking APIs.}
Published in 2008, Universal Plug and Play (UPnP) is another widely-deployed set of networking protocols that facilitate discovery and interaction between devices on the same network, with minimal configuration.
Since it leverages common protocols (HTTP/XML/SOAP on UDP/IP) and is agnostic to the link medium, it is truly cross-platform, extending not only to phones and laptops but also to printers, WiFi routers, and audio-visual equipment, to name a few examples.
UPnP has been characterized as consisting of protocols that are more specific to particular classes of devices and applications; this is in contradistinction to Zeroconf, which aims to provide a device-agnostic foundation on which any device class or application-level protocol can build.~\footnote{\url{http://www.zeroconf.org/zeroconfandupnp.html}}

More recently in 2017, as part of its \textit{Nearby} project, \textit{Google} released its \textit{Connections} API, which enables Android devices in close proximity to one another to communicate in a peer-to-peer fashion.~\footnote{\url{https://developers.google.com/nearby/connections/overview}, accessed 2017-12-11}
This is done over a seamless mix of Bluetooth and WiFi hotspots.
Unfortunately, Google has only made this API available on the Android platform; we seek a solution that is truly cross-platform.

\textbf{In-browser web servers.}
An early example of fully in-browser web server technology was \textit{Opera Unite} (2009)\footnote{\url{http://help.opera.com/Windows/12.10/en/unite.html, accessed 2017-12-11}}.
This extension to the Opera browser enabled users to serve general-purpose Web applications directly from their browsers. 
Communication between clients and servers in this way could either be via Opera Unite's proxy servers, which also offered a name registering and directory service, or direct peer-to-peer, the latter requiring more advanced technical configuration.
Importantly, Opera Unite applications were constrained to be written as Opera ``Widgets''\footnote{An Opera Widget is an application that runs on Opera's now-deprecated Widget Engine, which allows such apps to be run independently of the browser.}.
Though the service was popular with a sizeable subset of Opera users, especially for purposes such as file sharing and media streaming, Opera eventually retired Unite in 2012 to consolidate the multiple extension frameworks offered in its browser.\footnote{http://www.instantfundas.com/2012/04/opera-to-discontinue-unite-widgets-and.html}

% TODO (PTC): Possibly add some notes on Web Server Chrome, and PeerServer.

\section{Conclusion}
\label{sec:conclusion}

As networking capabilities become more ubiquitous across different types of devices, applications that communicate over local area networks become more common.
Using a proper suite of networking protocols and technologies, these applications can discover other devices in the network and exchange data with them.
One appealing way to build such apps is using zeroconf networks and web browsers.
Nonetheless, applications operating in zeroconf still face normal challenges imposed by the type of network that they operate on, e.g. 
unstable wireless links or host mobility. Also, developers have little to no browser API support for these technologies.


To address the aforementioned issues, we proposed \APINameNoSpace, which abstracts and decouples zeroconf logic from the application logic and seamlessly integrates fault-tolerance to applications that require a state but face failures due to intermittent connectivity. We evaluated \APIName by measuring network traffic and state recovery in face of failures in proof of concept applications. \APIName supports all its intended use cases and future work will improve bottlenecks identified during our evaluation.


\bibliographystyle{abbrv}
\bibliography{flyweb_paper}

\end{document}


\subsection{Methodology}
\label{sub:eval:evaluation_methodology}

In order to evaluate \APINameNoSpace, we created a small mobile network using an Android Pixel 2 smartphone mobile Wi-Fi hotspot.
We aimed to simulate intended uses cases where~\APIName could be used in practice. 
For instance, a group of executives could use their smartphones and share a business presentation with their peers, or teaching assistants could use their phones and start a queue application, as previously explained in our motivating example (Section~\ref{sec:motivating_example}).


We connected to our ad-hoc network with three MacOS computers, among which one acted as the starting server launching an instrumented version of a queue application built using the~\APIName API. 
All devices used Firefox Aurora Nightly build {\texttt{50.0a2}} and the three researchers acted as users of the queue application.


Throughout our evaluation we followed a small script describing tasks to be executed at each of the devices. 
After the server was started, each device
{\it (1)} discovered the running server in the network,
{\it (2)} connected to the server and
{\it (3)} executed either an enqueue or dequeue operation.
Afterwards,  
{\it (4)} the running server disconnected and 
{\it (5)} the successor node started a new server.
Finally, 
{\it (6)} all existing clients (re-)connected to the new server.

At step {\it (3)} we added an extra field to the payload of the called operations that contained a random string with variable size drawn from a long-tailed power-of-two distribution. 
By adding this field, we aimed to investigate the effects of different payloads/state sizes to \APIshort apps.
Our scripted execution was performed in a section that lasted about 40 minutes and some devices had one or more connections to the server node through multiple tabs.


Table~\ref{tbl:eval:experiment_summary} summarizes the events observed in our evaluation.
We had a minimum of 1 to a maximum of 8 devices connected at a given time, with 3 devices connected on average (sd $\rpm 1$ devices). 
There were 29 server disconnections and sequential server recovery events as well as a total of  28 executed operations. 
The total number of client connections represents the number of times a device either connected to a server or reconnected to a new server after a failure.

\begin{table}
    \caption{Experiment Summary}
    \label{tbl:eval:experiment_summary}
    \centering
    \begin{small}
    \begin{tabular}{C{2.5cm}|C{2.5cm}|C{2.5cm}}
    \hline
    \bfseries \# Server Recoveries & \bfseries \# Client Connections & \bfseries \# Operation calls \\
    \hline
    29 & 257 & 28 \\
    \hline
    \end{tabular}
    \end{small}
\end{table}

For each of the events presented in the Table~\ref{tbl:eval:experiment_summary}, we are interested in measuring the time between the occurrence of the event and the expected function calls in response to this event.
Each logged event is a tuple that contains at least an event name, device unique identifier, and a timestamp representing the milliseconds elapsed since the \texttt{UNIX} epoch.


For example, whenever a running server disconnects, our instrumented app logs a \texttt{disconnecting} event which is followed by a successor device responding with \texttt{become\_server\_begin} event and either a \texttt{become\_server\_end} or \texttt{become\_server\_error} event after the function executed. 
These events are chronologically sorted, as depicted in Listing~\ref{lst:events:server-recovery}, and we compute the elapsed time between them.

\begin{tiny}
\begin{lstlisting}[caption={Tuples with logged events},label={lst:events:server-recovery}, language=JavaScript]
(timestamp, event, deviceID, server)
(1512766249862, "disconnecting", 15127210, true),
(1512766265059, "become_server_begin", 15223107, false),
(1512766265669, "become_server_end", 15223107, true),
...
\end{lstlisting}    
\end{tiny}


Each device stored its own log using a local server and, after execution, all these files were grouped and processed to compute elapsed times. 
We chronologically sorted all collected events and filtered events of interest to compute our measurements. 
Results are discussed in the following subsections and we explicitly detail the events used to compute measurements for each subsection whenever necessary.




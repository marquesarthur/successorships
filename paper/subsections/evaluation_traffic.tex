\subsection{Traffic}
\label{sub:eval:traffic}

Finally, we evaluate network traffic in \APIshort applications.
We are interested in the round trip time (RTT) of messages as well as in how long it takes for client nodes to converge to a consistent state.
In order to compute RTT, we measured events related to devices executing the \texttt{Shippy.call} operation until they received a broadcast with a state update from the server node.
As each state update carries the most recent state version, we measure consistency as the time between the execution of \texttt{Shippy.call} and the last node updating to the most recent version.
For instance, if a server \texttt{A} currently in version \texttt{V1} has 3 successors \texttt{[B, C, D]}, and \texttt{B} calls the \texttt{add} operation, RTT is the time it takes for \texttt{B} to receive a state update.
On the other hand, consistency is longest time frame from the \texttt{add} call until \texttt{B}, \texttt{C}, or \texttt{D} updates to \texttt{V2}.


Figure~\ref{fig:message-RTT} plots the cumulative empirical function for RTT while Figure~\ref{fig:state-convergence} plots state convergence time.
In most cases, RTT is considerably fast and clients receive a state update in the order of milliseconds.
However, state convergence takes longer and in 90\% of the cases client nodes converge upon a new state roughly under 5 seconds (average of 1.3 seconds, sd $\rpm 2.7$ seconds).
We also noticed that results from the empirical cumulative function are affected by 4 outliers that are greater than 1.6 seconds. We hypothesize once again that the hotspot connection was not stable during convergence for these cases. 


Since state convergence takes longer than message RTT, we investigate how it is influenced by the number of connected nodes and payload sizes.
We followed a similar strategy as for client connection, evaluating boxplots per number of successors and convergence time as a linear function of the operation's payload.
For the sake of brevity we refrain from plotting charts for these scenarios.
It suffices to say that we could not see bottlenecks for either the number of nodes or the payload size. 
For example, using a payload with a size equal to the one that slew down client connection (1MB), convergence time was $\approx 1.6s$ while client connection time would be $\approx 20s$.
Therefore, we believe that \APIshort applications should not suffer considerable traffic bottlenecks and provide good user experience. 

\begin{figure*}
    \minipage{0.32\textwidth}%
        \centering
        \includegraphics[width=0.9\textwidth]{client-welcome-state-size}
        \caption{Client time to connect as a function of the server state size}\label{fig:client-welcome-state-size}
    \endminipage\hfill
    \minipage{0.32\textwidth}
        \centering
        \includegraphics[width=0.9\textwidth]{message-RTT}
        \caption{Operation Round Trip Time}
        \label{fig:message-RTT}
    \endminipage\hfill
    \minipage{0.32\textwidth}%
        \centering
        \includegraphics[width=0.9\textwidth]{state-convergence}
        \caption{State convergence - time it takes for all clients to update their state}
        \label{fig:state-convergence}
    \endminipage
\end{figure*}